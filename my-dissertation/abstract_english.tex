% Abstract
This work examines the potential of reducing carbon emissions by optimizing the pasta cooking process. Utilizing advanced natural language processing techniques, we analyzed traditional and innovative cooking methods to assess their energy consumption and environmental impact. The results indicate that adopting intelligent cooking methods can significantly lower energy use and reduce greenhouse gas emissions. Additionally, the paper explores the importance of public awareness and behavioral changes in promoting eco-friendly cooking practices. The study demonstrates that by adjusting everyday cooking habits, individuals can contribute to combating global warming on a micro-level, achieving sustainable development goals. This research offers new perspectives and practical guidelines for future environmentally conscious dietary practices.
